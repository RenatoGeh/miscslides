\documentclass{beamer}

\usepackage[brazilian]{babel}
\usepackage{hyperref}
\usepackage{multicol}

\setbeamercovered{transparent}

\makeatletter
\def\blfootnote{\xdef\@thefnmark{}\@footnotetext}
\makeatother

\title{\LARGE Pós-Graduação em Ciência da Computação}
\subtitle{\Large Informações sobre o programa}
\author{Renato Lui Geh\footnote{\url{renatolg@ime.usp.br}}\\Nathan Proença
  \footnote{\url{nathan@ime.usp.br}}}
\date{}

\begin{document}

\maketitle

\begin{frame}
  \frametitle{Representante Discente (RD)}

  \textbf{Formalmente:}
  \begin{itemize}
    \item Voz do corpo discente no Instituto
    \item Ponte entre docentes e discentes
    \item Representação em comissões
  \end{itemize}~\\\pause

  \textbf{Informalmente:}
  \begin{itemize}
    \item Apoio acadêmico
    \item Apoio emocional
    \item Organização do corpo estudantil
  \end{itemize}~\\

  Quem são os Representantes Discentes (RDs)?
\end{frame}

\begin{frame}
  \frametitle{Representantes Discentes no IME}

  \textbf{Conselho do Departamento de Computação}
  \begin{itemize}
    \item Renato Lui Geh (eu!)
  \end{itemize}~\\\pause

  \textbf{Comissão Coordenadora do Programa de Pós-Graduação em Ciência da Computação (CCPCC)}
  \begin{itemize}
    \item Renato Lui Geh (eu!), Nathan Proença
  \end{itemize}~\\\pause

  \textbf{Congregação}
  \begin{itemize}
    \item Fernando Lima, Ana Luíza Tenório
  \end{itemize}~\\\pause

  \textbf{Comissão de Pós-Graduação (CPG)}
  \begin{itemize}
    \item Ricardo Canale
  \end{itemize}~\\ 
\end{frame}

\begin{frame}
  \frametitle{Comissão Coordenadora do Programa (CCP)}

  \textbf{Qual o papel da CCP?}
  \begin{itemize}
    \item Aprovar bancas (quali ou defesas)
    \item Aprovar exames de proficiência
    \item Definir e atualizar as normas do programa
    \item Distribuição de bolsas
    \item Desligamentos e problemas
    \item Mudança de orientador
    \item Trancamentos e aproveitamentos de crédito
  \end{itemize}~\\\pause

  \textbf{Composição dos membros}
  \begin{itemize}
    \item 5 professores titulares e 5 suplentes
    \begin{itemize}
      \item Presidente: Prof. Dr. Alfredo Goldman
      \item Vice-presidente: Profa. Dra. Leliane Barros
    \end{itemize}
    \item 1 RD titular e 1 suplente
  \end{itemize}
\end{frame}

\begin{frame}
  \frametitle{Submetendo documentos para a CCP}

  \small
  \textbf{Passos para submissão}
  \begin{enumerate}\footnotesize
    \item Converse com o orientador antes de submeter!
    \item Preencha o formulário adequado (ver abaixo)
    \item Submeta para a secretaria da CCP
    \item Sua submissão será avaliada na próxima reunião da CCP
  \end{enumerate}
  Email da secretária da CCP (Katia): \url{secccpcomp@ime.usp.br}\pause

  \textbf{\underline{Sempre verifique as normas antes!}}
  \begin{itemize}\footnotesize
    \item Gerais: {\tiny\url{https://www.ime.usp.br/dcc/pos/normas}}
      \item Teses e dissertações: {\tiny\url{https://www.ime.usp.br/dcc/pos/normas/tesesedissertacoes}}
  \end{itemize}\pause

  \textbf{Formulários:} {\tiny\url{https://www.ime.usp.br/dcc/pos/formularios}}

  \textbf{Calendário}
  \begin{itemize}\footnotesize
    \item Acadêmico: {\tiny\url{tinyhttps://www.ime.usp.br/dcc/pos/calendarioescolar}}
    \item Reuniões CCP: {\tiny\url{https://www.ime.usp.br/dcc/pos/calendarioreunioes}}
  \end{itemize}\pause

  \textbf{\underline{Fiquem atentos às datas das reuniões da CCP!}}
\end{frame}

\begin{frame}
  \frametitle{Linhas de pesquisa}

  \footnotesize
  \begin{itemize}
    \item Laboratório TACO (Theory, Algorithms and Combinatorial Optimization)
      \begin{itemize}\scriptsize
        \item Otimização Combinatória
      \end{itemize}\pause
    \item Laboratório CompMus + OtiCon
      \begin{itemize}\scriptsize
        \item Computação Musical
        \item Otimização Contínua
      \end{itemize}\pause
    \item Laboratória E-Science + Imagem
      \begin{itemize}\scriptsize
        \item Data Science
        \item Visão Computacional e Processamento de Imagem
      \end{itemize}\pause
    \item Laboratório de Sistemas
      \begin{itemize}\scriptsize
        \item Criptografia e Segurança
        \item Educação
        \item Sistemas e Redes
      \end{itemize}\pause
    \item Laboratório LIAMF (Lógica, Inteligência Artificial e Métodos Formais)
      \begin{itemize}\scriptsize
        \item Inteligência Artificial e Lógica
      \end{itemize}
  \end{itemize}

  \small Como fazer para trocar sua pesquisa?
\end{frame}

\begin{frame}
  \frametitle{Trocando de orientador}

  \textbf{Como fazer?}
  \begin{enumerate}
    \item Converse com o seu atual orientador ou orientadores
    \item Converse com o orientador que deseja trocar
    \item Orientador atual, assim como futuro precisam assinar o formulário
    \item Submeta formulário na secretaria para aprovação da CCP
  \end{enumerate}~\\\pause

  \textbf{Não fique com medo de trocar caso esteja infeliz!}\\~\\

  Lista de orientadores com áreas de pesquisa:
  \begin{itemize}
    \item \url{https://www.ime.usp.br/dcc/pos/areas}
  \end{itemize}
\end{frame}

\begin{frame}
  \frametitle{Matrícula}

  \textbf{Via Janus (\url{https://uspdigital.usp.br/})}
  \begin{itemize}
    \item Abre antes de cada semestre
    \item Se tem bolsa, tem que cadastrar/atualizar todo semestre
    \item Matrícula $\to$ Pré-matrícula
  \end{itemize}~\\\pause

  \textbf{Período de retificação e cancelamento (por disciplina)}
  \begin{description}[Cancelamento]
    \item[Retificação:]~\\
      \begin{itemize}
        \item Mudar em que disciplinas você está matriculado
        \item Ocorre logo no começo do semestres
      \end{itemize}\pause
    \item[Cancelamento:]~\\
      \begin{itemize}
        \item Cancelar uma matrícula após o início do semestre letivo
        \item Prazo varia por disciplina
      \end{itemize}
  \end{description}
\end{frame}

\begin{frame}
  \frametitle{Matrícula}

  \textbf{Matrícula de acompanhamento}
  \begin{itemize}
    \item Após terminar créditos necessários
    \item Matrícula $\to$ Solicitação de matrícula de acompanhamento
    \item Pode resultar em desligamento se não se matricular!!!
  \end{itemize}~\\

  \textbf{Cancelando disciplina}
  \begin{itemize}
    \item Matrícula $\to$ Cancelamento
    \item Prazo é normalmente até 50\% do período da disciplina
  \end{itemize}~\\

  \textbf{Fique atento, o prazo fica no oferecimento da disciplina!}\\~\\
  \centering\includegraphics[width=0.5\textwidth]{imgs/cancelamento.png}
\end{frame}

\begin{frame}
  \frametitle{Disciplinas obrigatórias}

  \textbf{Quantas?}
  \begin{itemize}
    \item 1 de Teoria e 1 de Sistemas para Mestrado/Doutorado
    \item 2 de Teoria e 2 de Sistemas para Doutorado Direto
  \end{itemize}~\\\pause

  \textbf{Quais valem?}
  \begin{itemize}
    \item Ver as normas gerais
    \item Introdução a Teoria dos Grafos conta como obrigatória de Teoria mas não consta nas normas
  \end{itemize}~\\\pause

  \textbf{Tem prazo?}\pause Apenas para quem tem bolsa institucional
  \begin{itemize}
    \item 12 meses para Mestrado/Doutorado
    \item 18 meses para Doutorado Direto
  \end{itemize}
\end{frame}

\begin{frame}
  \frametitle{Bolsa institucional}

  \textbf{Para quem tem bolsa institucional}
  \begin{itemize}
    \item Fique atento às regras de manutenção
    \item \url{https://www.ime.usp.br/dcc/pos/bolsas}
    \item Regras muito complicadas, leia com atenção\pause
    \item Na prática:
  \end{itemize}

  \begin{multicols}{2}
    \textbf{Mestrado}
    \begin{description}[24 meses]
      \item[6 meses:] 24 créditos
      \item[12 meses:] 24+20 créditos
      \item[18 meses:] qualificação
      \item[24 meses:] defesa
    \end{description}~\\

    \textbf{Doutorado}
    \begin{description}[48 meses]
      \item[6 meses:] 16 créditos
      \item[12 meses:] 16+16 créditos
      \item[18 meses:] 16+16+12 créditos
      \item[30 meses:] qualificação
      \item[48 meses:] defesa
    \end{description}
  \end{multicols}
\end{frame}

\begin{frame}
  \frametitle{Créditos alternativos}
  \textbf{São 44 créditos obrigatórios.} Disciplinas normalmente são 8 créditos.\\~\\\pause

  \textbf{4 créditos (no máximo) de créditos alternativos.}
  \begin{multicols}{2}
  \begin{description}[Cursos de verão]
    \item[Monitorias PAE:] 2 créditos
    \item[Artigos:] 2 créditos\columnbreak
    \item[Disciplinas USP:] depende
    \item[Cursos de verão:] depende
  \end{description}\pause
  \end{multicols}

  \textbf{Monitoria PAE}
  \begin{itemize}
    \item Obrigatório para quem tem bolsa de Doutorado da CAPES
    \item Precisa fazer antes:
      \begin{itemize}
        \item Série de seminários (semestres ímpares); ou
        \item Disciplina GEN5711 (4 créditos)
      \end{itemize}
    \item Paga um pouco mais que a Monitoria IME.
    \item Precisa fazer cronograma e relatório, junto com acompanhamento
  \end{itemize}
\end{frame}

\begin{frame}
  \frametitle{Exame de proficiência (Inglês)}

  \textbf{Prazo}
  \begin{itemize}
    \item Vide Janus
  \end{itemize}~\\\pause

  \textbf{Inscrições}
  \begin{itemize}
    \item Fiquem atentos aos emails institucionais!
    \item Taxa de R\$80,00
  \end{itemize}~\\\pause

  \textbf{Avaliação}
  \begin{itemize}
    \item Feita pela FFLCH
  \end{itemize}~\\\pause

  \textbf{Alternativas}
  \begin{multicols}{2}
  \begin{itemize}
    \item TOEFL
    \item IELTS\columnbreak
    \item Cambridge
    \item Ver normas
  \end{itemize}
  \end{multicols}
\end{frame}

\begin{frame}
  \frametitle{Exame de proficiência (Português)}

  \textbf{Prazo}
  \begin{itemize}
    \item Vide Janus
  \end{itemize}~\\

  \textbf{Apenas para estrangeiros.}\\~\\

  \textbf{Alternativas}
  \begin{itemize}
    \item Disciplina ``Português para Estrangeiros'' na FFLCH
  \end{itemize}
\end{frame}

\end{document}
