\documentclass[usenames,dvipsnames]{beamer}

\usepackage[brazilian]{babel}
\usepackage{hyperref}
\usepackage{mathtools}
\usepackage{lyluatex}
\usepackage{realbookchords}
\usepackage{xcolor}

\usetheme{boxes}
\setbeamertemplate{blocks}[rounded][shadow]

\newcommand\numo[1]{#1\textsuperscript{o}}
\newcommand\numa[1]{#1\textsuperscript{a}}

\newcommand\srbc[1]{{\footnotesize\rbc{#1}}}

\newtheorem{definicao}{Definição}

\title{Improvisando com a escala pentatônica\\~\\\small ou: Como fingir que sei música com apenas 5
notas}
\date{}
\author{Renato Lui Geh}

\begin{document}

\maketitle

\begin{frame}
  \frametitle{Conteúdo}
  \tableofcontents
\end{frame}

\section{Introdução}

\begin{frame}
  \frametitle{Terminologia}

  Qual a diferença entre ``melodia'' e ``harmonia''?\\~\\

  \begin{itemize}
    \item \textbf{Melodia:} a parte ``cantável'';\\
    \item \textbf{Harmonia:} o ``acompanhamento''.
  \end{itemize}~\\

  Bazaluk mostrou \emph{Harmonia}. Eu vou mostrar \emph{Melodia}. $\ddot\smile$\\~\\

  Improvisação é composição musical em ``tempo-real''.
\end{frame}

\section{Identificando o centro tonal}

\begin{frame}
  \frametitle{Campo harmônico maior}

  \begin{block}{Definição. (Campo harmônico maior)}
    Seja $\Sigma$ um tom. O campo harmônico maior (CHM) de $\Sigma$ são 7 acordes cujas raízes
    (primeira nota do acorde) pertencem a cada nota da escala maior de $\Sigma$ de forma que sigam
    a seguinte progressão:
    \begin{center}
      \srbc{Ima7}\quad\srbc{IIm7}\quad\srbc{IIIm7}\quad\srbc{IVma7}\quad\srbc{V7}\quad\srbc{VIm7}\quad\srbc{VIIm7(\b5)}
    \end{center}
  \end{block}
  \pause

  \begin{exampleblock}{Exemplo. (Campo harmônico maior de \srbc{C})}
    \begin{center}
      \srbc{Cma7}\quad\srbc{Dm7}\quad\srbc{Em7}\quad\srbc{Fma7}\quad\srbc{G7}\quad\srbc{Am7}\quad\srbc{Bm7(\b5)}
    \end{center}
  \end{exampleblock}
  \pause

  \begin{exampleblock}{Exemplo. (Campo harmônico maior de \srbc{G})}
    \begin{center}
      \srbc{Gma7}\quad\srbc{Am7}\quad\srbc{Bm7}\quad\srbc{Cma7}\quad\srbc{D7}\quad\srbc{Em7}\quad\srbc{F\k m7(\b5)}
    \end{center}
  \end{exampleblock}
\end{frame}

\begin{frame}
  \frametitle{Identificando o centro tonal}

  \begin{alertblock}{Rule of thumb.}
    Se o tom/centro tonal é $\Sigma$, então acordes (possivelmente) seguirão o \emph{campo
    harmônico} de $\Sigma$.
  \end{alertblock}~\\
  \pause

  \begin{table}
    \begin{tabular}{lcccr}
      Dado & $\Sigma$ & $\to$ & Acorde $\in$ CHM$(\Sigma)$ & \textcolor{blue}{OK!}\\
      Dado & Acorde $\in$ CHM$(\Sigma)$ & $\to$ & $\Sigma$ & \textcolor{red}{???}
    \end{tabular}
  \end{table}~\\
  \pause

  \begin{alertblock}{Importante.}
    O acorde dominante \srbc{V7} (normalmente) define o centro tonal!\\
    Dado um acorde dominante, ``conte'' até 4 e você chegará à raiz (centro tonal)!
  \end{alertblock}
  \pause

  \begin{exampleblock}{Exemplo.}
    O acorde dominante \srbc{D7} é o quinto acorde do campo harmônico maior de \srbc{G}.
  \end{exampleblock}
\end{frame}

\begin{frame}
  \frametitle{Autumn Leaves - Parte 1}

  \lilypondfile{scores/autumnleaves_1.ly}
\end{frame}

\begin{frame}
  \frametitle{A escala menor}

  \begin{alertblock}{Dica.}
    Se a \srbc{V7} estabelece um \srbc{Im7}, então estamos no campo harmônico menor.
  \end{alertblock}
  \pause

  \begin{exampleblock}{Exemplo.}
    A dominante \srbc{B7} estabelece o centro tonal de \srbc{Em7} em Autumn Leaves.
  \end{exampleblock}
  \pause

  \begin{alertblock}{Dica.}
    A escala menor de $M$ está a $1\frac{1}{2}$ tom de distância abaixo da sua escala maior
    relativa.
  \end{alertblock}
  \pause

  \begin{exampleblock}{Exemplo.}
    A escala menor de \srbc{Em} tem como relativa maior \srbc{Gma}.
  \end{exampleblock}
\end{frame}

\begin{frame}
  \frametitle{Autumn Leaves - Parte 2}

  \lilypondfile{scores/autumnleaves_2.ly}
\end{frame}

\section{Escala pentatônica}

\begin{frame}[fragile]
  \frametitle{Escala pentatônica}

  \begin{block}{Definição. (Escala pentatônica maior)}
    É a escala composta pela \numa{1}, \numa{2}, \numa{3}, \numa{5} e \numa{6} notas da escala maior.
  \end{block}~\\
  \pause

  Para a escala maior de \rbc{C}, teríamos:\\

  \begin{center}
    \begin{lilypond}
      \include "lilyjazz.ily"
      \include "jazzchords.ily"
      \include "jazzextras.ily"

      \score {
        \relative c' {
          \time 5/4
          \clef treble
          c4 d4 e4 g4 a4
        }
      }
    \end{lilypond}
  \end{center}~\\
  \pause

  Para escala maior de \rbc{G} seria:\\

  \begin{center}
    \begin{lilypond}
      \include "lilyjazz.ily"
      \include "jazzchords.ily"
      \include "jazzextras.ily"

      \score {
        \relative g' {
          \key g \major
          \time 5/4
          \clef treble
          g4 a4 b4 d4 e4
        }
      }
    \end{lilypond}
  \end{center}
\end{frame}

\section{Escala blues}

\begin{frame}[fragile]
  \frametitle{Escala blues}

  \begin{block}{Definição. (Escala blues maior)}
    É a escala composta pela pentatônica maior + blue note (3\srbc{\b}):
    \begin{center}
      1, 2, 3\srbc{\b}, 3, 5, 6.
    \end{center}
  \end{block}~\\
  \pause

  Para \rbc{C}, teríamos:\\

  \begin{center}
    \begin{lilypond}
      \include "lilyjazz.ily"
      \include "jazzchords.ily"
      \include "jazzextras.ily"

      \score {
        \relative c' {
          \time 6/4
          \clef treble
          c4 d ees e g a
        }
      }
    \end{lilypond}
  \end{center}~\\
  \pause

  Para \rbc{G} seria:\\

  \begin{center}
    \begin{lilypond}
      \include "lilyjazz.ily"
      \include "jazzchords.ily"
      \include "jazzextras.ily"

      \score {
        \relative g' {
          \key g \major
          \time 6/4
          \clef treble
          g4 a ais b d e
        }
      }
    \end{lilypond}
  \end{center}
\end{frame}

\section{A quinta aumentada}

\begin{frame}[fragile]
  \frametitle{Quinta aumentada}

  \begin{block}{Definição. (Escala blues maior + 5\srbc{\s})}
    Escala blues maior + quinta aumentada (5\srbc{\s}): 1, 2, 3\srbc{\b}, 3, 5, 5\srbc{\s}, 6.
  \end{block}
  \pause

  Estamos ``emprestando'' uma nota do campo harmônico menor.\\~\\

  Para \rbc{C}, teríamos:\\

  \begin{center}
    \begin{lilypond}
      \include "lilyjazz.ily"
      \include "jazzchords.ily"
      \include "jazzextras.ily"

      \score {
        \relative c' {
          \time 7/4
          \clef treble
          c4 d ees e g gis a
        }
      }
    \end{lilypond}
  \end{center}~\\
  \pause

  Para \rbc{G} seria:\\

  \begin{center}
    \begin{lilypond}
      \include "lilyjazz.ily"
      \include "jazzchords.ily"
      \include "jazzextras.ily"

      \score {
        \relative g' {
          \key g \major
          \time 7/4
          \clef treble
          g4 a ais b d dis e
        }
      }
    \end{lilypond}
  \end{center}
\end{frame}

\begin{frame}
  \begin{center}
    \large Obrigado!\\~\\
    \normalsize Slides e código disponíveis em \url{https://github.com/RenatoGeh/miscslides}
  \end{center}
\end{frame}

\end{document}
